\documentclass[brazil]{beamer}
\usepackage{beamerthemesplit}
\usepackage[brazilian]{babel}
\usepackage[utf8]{inputenc}
\usepackage{color}
\usepackage{xcolor}
\usepackage{graphicx}
%\usepackage{subcaption}
\usepackage{float}
\usepackage{wrapfig}
\usepackage{amssymb}
\usepackage{amsmath}
\usepackage{fancybox}
\usepackage{ulem}
\usepackage{listings}
\usepackage{upquote}
\usetheme{JuanLesPins}
%\usetheme{Warsaw}
%\usetheme{CambridgeUS}
%\usetheme{Malmoe}


%\newcommand{\lyxline}[1][1pt]{%
%  \par\noindent%
%  \rule[.5ex]{\linewidth}{#1}\par}


\title{
  Etherclan
}
\subtitle{
  DESCRIÇÃO
}
\author{Henrique Gemignani Passos Lima}

\begin{document}

% ---------------------------------------------------------------------------- %
% Opções de listing usados para o código fonte
% Ref: http://en.wikibooks.org/wiki/LaTeX/Packages/Listings
\lstdefinelanguage{lua}
  {morekeywords={and,break,do,else,elseif,end,false,for,function,if,in,local,
     nil,not,or,repeat,return,then,true,until,while},
   morekeywords={[2]arg,assert,collectgarbage,dofile,error,_G,getfenv,
     getmetatable,ipairs,load,loadfile,loadstring,next,pairs,pcall,print,
     rawequal,rawget,rawset,select,setfenv,setmetatable,tonumber,tostring,
     type,unpack,_VERSION,xpcall},
   morekeywords={[2]coroutine.create,coroutine.resume,coroutine.running,
     coroutine.status,coroutine.wrap,coroutine.yield},
   morekeywords={[2]module,require,package.cpath,package.load,package.loaded,
     package.loaders,package.loadlib,package.path,package.preload,
     package.seeall},
   morekeywords={[2]string.byte,string.char,string.dump,string.find,
     string.format,string.gmatch,string.gsub,string.len,string.lower,
     string.match,string.rep,string.reverse,string.sub,string.upper},
   morekeywords={[2]table.concat,table.insert,table.maxn,table.remove,
   table.sort},
   morekeywords={[2]math.abs,math.acos,math.asin,math.atan,math.atan2,
     math.ceil,math.cos,math.cosh,math.deg,math.exp,math.floor,math.fmod,
     math.frexp,math.huge,math.ldexp,math.log,math.log10,math.max,math.min,
     math.modf,math.pi,math.pow,math.rad,math.random,math.randomseed,math.sin,
     math.sinh,math.sqrt,math.tan,math.tanh},
   morekeywords={[2]io.close,io.flush,io.input,io.lines,io.open,io.output,
     io.popen,io.read,io.tmpfile,io.type,io.write,file:close,file:flush,
     file:lines,file:read,file:seek,file:setvbuf,file:write},
   morekeywords={[2]os.clock,os.date,os.difftime,os.execute,os.exit,os.getenv,
     os.remove,os.rename,os.setlocale,os.time,os.tmpname},
   alsodigit = {.},
   sensitive=true,
   morecomment=[l]{--},
   morecomment=[s]{--[[}{]]},
   morestring=[b]",
   morestring=[d]',
   morestring=[s]{[[}{]]},
  }

\frame{
  \titlepage
  Orientador: Prof. Dr. Daniel Macedo Batista
}

\frame{\tableofcontents}

%-------------------------------------
\section{Introdução: Motivações e Objetivos}
%-------------------------------------
\frame{
  \begin{center}
  \LARGE 1. Introdução: Motivações e Objetivos
  \end{center}
}
%-------------------------------------
\begin{frame}[fragile]
  \frametitle{Motivações}
  \pause
  %\begin{figure}
  %  \includegraphics[width=.9\textwidth]{images/horus+sublime.png}
  %\end{figure}
  \vspace{-10pt}
  \begin{itemize}
    \pause
    \item STUFF
  \end{itemize}
\end{frame}
%-------------------------------------
\frame{
  \frametitle{Proposta de TCC (2013)}
  \pause
  \vspace{-20pt}
  \begin{center}
    Etherclan
  \end{center}
  \vspace{20pt}
  \begin{itemize}
    \pause
    \item STUFF
  \end{itemize}
}
%-------------------------------------
%\section{Por trás de uma linguagem de script}
%-------------------------------------
\begin{frame}[fragile]
  \frametitle{STUFF}
  \pause
  \begin{block}{myscript.lua}
    \begin{lstlisting}[language=lua]
variable = 42
function foo (x, y)
  return x+y
end
    \end{lstlisting}
  \end{block}
\end{frame}
%-------------------------------------
%\section{Unlimited slide works}
%-------------------------------------
\begin{frame}
  \begin{center}
    \LARGE OK!
  \end{center}
\end{frame}
%-------------------------------------
\begin{frame}
  \frametitle{Bibliografia}
  \begin{itemize}
    \footnotesize
    \item[1]
      %Lua C API. http://www.lua.org/manual/5.1/manual.html\#5, Novembro 2013.
    \vspace{1em}
  \end{itemize}
\end{frame}
%-------------------------------------
\end{document}

