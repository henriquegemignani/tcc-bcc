\chapter*{Parte Objetiva}
\label{sec:parte_objetiva}
%% --------------------------------------------------------------------- %%
\chapter{Introdução}
\label{sec:intr}

Desenvolvedores de jogos sempre tentaram introduzir formas de que diversas
pessoas possam se divertir com um jogo simultaneamente. Em jogos como o Pong
clássico, a tela era compartilhada e tinha controles separados para cada 
jogador. Outra opção, era dividir a tela em duas (ou mais), uma para cada
jogador. Tais soluções eram viáveis em jogos para consoles, em que era comum
poder utilizar dois ou mais controles, mas jogos parfa computadores normalmente
usam teclado e mouse, periféricos que raramente existem em duplicidade.
Apesar disso, o que é de fato comum no ambiente de computadores, é cada um ter
a sua própria máquina. Com isso em mente, jogos utilizavam de quaisquer recursos
de comunicação entre máquinas existentes, desde comunicação direta com modens
a redes locais IPX.

Com a popularidade da Internet e do TCP/IP, se tornou cada vez mais fácil
jogadores se conectarem remotamente. Comunidades se formaram para facilitar
encontrar outros jogadores, montar campeonatos, guildas e conversar. Empresas
adotaram essa ideia de comunidades, criando comunidades oficiais, com integração
direta com os jogos, como a Battle.net e Gamespy, facilitando todos os usuários 
e ainda permitindo novas funcionalidades, como \textit{matchmaking} baseado em
nível de habilidade e proteção anti-trapaças. No entanto, tais funcionalidades
dependiam de um servidor principal mantido pela empresa responsável pelo jogo.

Quando você está lidando com um projeto de um jogo eletrônico com um baixo
orçamento, como por exemplo um jogo independente ou um software livre, não é
factível manter os servidores necessários para sustentar um serviço de
multijogador online.

No capítulo 2 temos diversos conceitos para auxiliar na compreensão do texto,
no capítulo 3 apresentamos o resultado deste trabalho, a rede Etherclan, e como
que ela foi organizada. No capítulo 4 contém os experimentos realizados utilizando
a Etherclan, enquanto no capítulo 5 temos as conclusões. Os capítulos 6 e 7 

\section{Motivações}
\label{sec:intr:motivacoes}

Seria interessante que cada jogador fosse capaz de encontrar outros clientes
sem a necessidade de um servidor central, permitindo prover boa parte, se não
mesmo todas, as funcionalidades frequentemente encontradas em jogos comerciais.

Além disso, a possibilidade de não só expandir o jogo \textit{vikings}, mas como
desenvolver um modo multijogador é algo que sempre me interessou, e poder
realizar ambos como um projeto de faculdade me convenceram a escolher esse
projeto como o meu trabalho de conclusão de curso.

\section{Objetivos}
\label{sec:intr:objetivos}

Esse trabalho de conclusão de curso tem como objetivos:

\begin{enumerate}
  \item Desenvolver uma rede distribuída, incluindo definir o protocolo e uma implementação deste com capacidade de 
    buscar novos nós, armazenar os conhecidos e ter politicas para remover nós.
    
  \item Adaptação do jogo \textit{vikings} para utilizar recursos multijogador via rede, incluindo partidas
    multijogadores.
\end{enumerate}

\input cap-conceitos
\input cap-rede
\input cap-resultados
\input cap-dificuldades

