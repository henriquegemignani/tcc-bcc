\chapter*{Parte Objetiva}
\label{sec:parte_objetiva}
%% --------------------------------------------------------------------- %%
\chapter{Introdução}
\label{sec:intr}

Desenvolvedores de jogos sempre tentaram introduzir formas de que diversas
pessoas possam se divertir com um jogo simultaneamente. Em jogos como o Pong
clássico, a tela era compartilhada e tinha controles separados para cada
jogador. Outra opção, era dividir a tela em duas (ou mais), uma para cada
jogador. Tais soluções eram viáveis em jogos para consoles, que era comum poder
utilizar dois ou mais controles, mas jogos para computadores normalmente usam
teclado e mouse, o que não é comum ter mais de um par em uma máquina.
Apesar disso, o que é de fato comum no ambiente de computadores, é cada um ter
a sua própria máquina. Com isso em mente, jogos utilizavam de quaisquer recursos
de comunicação entre máquinas existentes, desde comunicação direta com modens
a redes locais IPX.

Com a popularidade da Internet e do TCP/IP, se tornou cada vez mais fácil
jogadores se conectarem remotamente. Comunidades se formaram para facilitar
encontrar outros jogadores, montar campeonatos

Quando 

\section{Motivações}
\label{sec:intr:motivacoes}

Interesse em fazer um jogo eletrônico com multijogador através da Internet, que não dependa de manter um servidor para sempre, mas que também não dependa de achar um endereço IP em comunidades mantidas por fãs.

TODO: o vikings

\section{Objetivos}
\label{sec:intr:objetivos}

\begin{enumerate}
  \item Uma rede distribuída, incluindo definir o protocolo e uma implementação deste com capacidade de 
    buscar novos nós, armazenar os conhecidos e ter politicas para remover nós.
    
  \item Adaptação do jogo \textit{vikings} para utilizar recursos multijogador via rede, incluindo partidas
    multijogadores.
\end{enumerate}

\input cap-conceitos
\input cap-resultados
\input cap-dificuldades

