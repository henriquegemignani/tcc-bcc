\chapter{Resultados}
\label{sec:resultados}

\section{Rede Distribuída}
\label{sec:resultados:rede}

  \subsection{Protocolo}
    \subsubsection{Anunciando novo nó}
      Sempre que um cliente cria um novo nó da rede, este envia para todos os nós que ele conhece, 
      a mensagem 'ANNOUNCE\_SELF' junto da porta onde escuta.
      
      O nó que recebe a mensagem responde 'KNOWN' ou 'WELCOME', dependendo se já conhecia o par
      (IP, PORTA) do nó novo.
      
      Como o cliente trata as duas respostas ainda não foi definido.
      
    \subsubsection{Pedindo novos nós}
      Um cliente pode mandar a mensagem 'REQUEST\_CLIENTS' para qualquer nó da rede para pedir clientes.
      
      A resposta para essa mensagem ainda não foi definido.
      
  \subsection{Implementação}
    A implementação do protocolo deste trabalho será feita em Lua usando o LuaSocket, %TODO links
    para poder ser utilizado com facilidade pelo \textit{vikings}, que é feito em Lua.
  
  \subsection{Problemas}
    \subsubsection{Bootstrap}
      Na descrição do comando \textit{Anunciando novo nó}, foi dito que essa mensagem é enviada a todos
      os nós conhecidos. Se é a primeira vez que rodamos o cliente numa máquina, quais são esses nós?
      
      Ainda não achei nenhuma solução ótima para o problema. Até então, o plano é que cada software
      que utiliza a rede tem uma lista fixa (\textit{hardcoded}) de nós inicias.
\section{Chat}
\label{sec:resultados:chat}

\section{Vikings Multijogador}
\label{sec:resultados:vikings}
