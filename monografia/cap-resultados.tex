\chapter{Resultados}
\label{sec:resultados}

\section{Rede Distribuída}
\label{sec:resultados:rede}

  A rede NOME é composta por diversos nós conectados diretamente um nos outros, sem a necessidade de nenhum servidor centralizado.
  
  Para um programa usufruir dos recursos da rede, é necessário primeiro fazer parte da rede. Para tal duas coisas são necessárias:
  \begin{itemize}
    \item Conhecer pelo menos 1 nó membro da rede.
    \item Criar um nó para inserir na rede.
  \end{itemize}
  
  \subsection{Um Nó}
    \subsubsection{O que é um nó}
      Todo nó deve possuir um UUIDv4 próprio.
      Para um nó ser considerado ativo, deve-se escutar em alguma porta TCP qualquer e responder às mensagens recebidas.
      
    \subsubsection{Processo de criação de um nó ativo}
      Deve-se gerar um UUIDv4, e criar um socket para dar listen.

    \subsubsection{Adicionar o nó na rede}
      Com o nó ativo, envie a mensagem 'ANNOUNCE\_SELF' para pelo menos 1 nó que faça parte da rede.

  \subsection{Mensagens do Protocolo}
    Mensagens são cadeias de caracteres codificadas em UTF-8, terminadas por uma quebra de linha
    (caractere ASCII '\\n'). Uma mensagem pode ter 0 ou mais ocorrencias do caractere ASCII ' ',
    dividindo a mensagem em um comando seguindo de 0 ou mais argumentos.
    
    \subsection{Comandos}
      Temos a seguir todos os comandos formais da rede, incluindo quais argmentos são esperados para cada um.
        
      \begin{itemize}
        \item ANNOUNCE\_SELF <UUID> <PORT> \\
          O IP de origem está anunciando que criou um novo nó, com o UUID enviado e que está escutando
          em seu IP, na porta dada.
          
          Esse comando tem duas respostas possíveis, 'KNOWN' e 'WELCOME', descritas na próxima seção.
            Sempre que um cliente cria um novo nó da rede, este envia para todos os nós que ele conhece, 
            a mensagem 'ANNOUNCE\_SELF' junto da porta onde escuta.
      
            O nó que recebe a mensagem responde 'KNOWN' ou 'WELCOME', dependendo se já conhecia o par
            (IP, PORTA) do nó novo.
      
        \item REQUEST\_NODES \\
          Um cliente pode mandar a mensagem 'REQUEST\_CLIENTS' para qualquer nó da rede para pedir clientes.
          
          A resposta para essa mensagem ainda não foi definido.
      \end{itemize}
      
    \subsection{Respostas}
      Temos a seguir todos as respostas formais da rede.
      
      \begin{itemize}
        \item KNOWN \\
          O par (IP, PORT) dado já era conhecido.
      
  \subsection{Implementação}
    A implementação do protocolo deste trabalho será feita em Lua usando o LuaSocket, %TODO links
    para poder ser utilizado com facilidade pelo \textit{vikings}, que é feito em Lua.
  
  \subsection{Problemas}
    \subsubsection{Bootstrap}
      Na descrição do comando \textit{Anunciando novo nó}, foi dito que essa mensagem é enviada a todos
      os nós conhecidos. Se é a primeira vez que rodamos o cliente numa máquina, quais são esses nós?
      
      Ainda não achei nenhuma solução ótima para o problema. Até então, o plano é que cada software
      que utiliza a rede tem uma lista fixa (\textit{hardcoded}) de nós inicias.
\section{Chat}
\label{sec:resultados:chat}

\section{Vikings Multijogador}
\label{sec:resultados:vikings}
