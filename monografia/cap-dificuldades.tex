\chapter{Conclusões}
\label{sec:conclusoes}

Com a implementação da rede e os dois protótipos de serviços, o que pode ser dito sobre a proposta?

Um ponto importante de notar é que a proposta funciona. É possível encontrar outros usuários
através da rede e trocar algumas mensagens. Mas infelizmente, as boas notícias terminam aí.
Subestimar a importância de critérios bem definidos para busca e gerenciamento de nós conhecidos
elimina qualquer esperança de que a rede tenha desempenho satisfatório em algum ambiente de
produção. 

Não só isso, mas burlar NATs, necessário já que a maioria das redes hoje em dia estão atrás de
roteadores com um único endereço IP válido, quando se utiliza TCP é bem mais complexo que usando
UDP (vide RFC5128 \cite{rfc5128}), além de que utilizar exclusivamente TCP remove a opção de poder
enviar pacotes 'leves' que o UDP permite.

Finalmente, além de tudo dito anteriormente, ao implementar o modo multijogador do vikings ficou
claro que a possibilidade de permitir que dois peers enviem e recebam mensagens arbitrárias
utilizando a infraestrutura do Etherclan é algo bem conveniente. Em outras palavras, oficializar
que o Etherclan é uma rede sobreposta (\textit{overlay network}) e explorar esse fato.

\section{Trabalhos Futuros}
\label{sec:conclusoes:futuro}

  Com tudo isso em mente, o que fazer daqui em diante?

  \begin{enumerate}
    \item Desenvolver uma implementação do Etherclan em C++, para facilitar a integração com outros
      projetos.
    \item Modificar o protocolo para utilizar UDP.
    \item Permitir que dois softwares troquem mensagens utilizando o Etherclan diretamente, tornando
      desnecessário o uso direto de sockets.
    \item Estudar e implantar algoritmos para realização de busca e armazenamento de nós.
  \end{enumerate}