% Arquivo LaTeX de exemplo de dissertação/tese a ser apresentados à CPG do IME-USP
% 
% Versão 5: Sex Mar  9 18:05:40 BRT 2012
%
% Criação: Jesús P. Mena-Chalco
% Revisão: Fabio Kon e Paulo Feofiloff
%  
% Obs: Leia previamente o texto do arquivo README.txt

% Alterado para monografia do trabalho de formatura (graduação).

\documentclass[11pt,twoside,a4paper]{book}

% ---------------------------------------------------------------------------- %
% Pacotes 
\usepackage[T1]{fontenc}
\usepackage[brazil]{babel}
\usepackage[utf8]{inputenc}
\usepackage[pdftex]{graphicx}           % usamos arquivos pdf/png como figuras
\usepackage{setspace}                   % espaçamento flexível
\usepackage{indentfirst}                % indentação do primeiro parágrafo
\usepackage{makeidx}                    % índice remissivo
\usepackage[nottoc]{tocbibind}          % acrescentamos a bibliografia/indice/conteudo no Table of Contents
\usepackage{courier}                    % usa o Adobe Courier no lugar de Computer Modern Typewriter
\usepackage{type1cm}                    % fontes realmente escaláveis
\usepackage{listings}                   % para formatar código-fonte (ex. em Java)
\usepackage{titletoc}
\usepackage{amsfonts}
\usepackage{amsmath}
\usepackage{clrscode}
%\usepackage[bf,small,compact]{titlesec} % cabeçalhos dos títulos: menores e compactos
\usepackage[fixlanguage]{babelbib}
\usepackage[font=small,format=plain,labelfont=bf,up,textfont=it,up]{caption}
\usepackage[usenames,svgnames,dvipsnames]{xcolor}
\usepackage[a4paper,top=2.54cm,bottom=2.0cm,left=2.0cm,right=2.54cm]{geometry} % margens
%\usepackage[pdftex,plainpages=false,pdfpagelabels,pagebackref,colorlinks=true,citecolor=black,linkcolor=black,urlcolor=black,filecolor=black,bookmarksopen=true]{hyperref} % links em preto
\usepackage[pdftex,plainpages=false,pdfpagelabels,pagebackref,colorlinks=true,citecolor=DarkGreen,linkcolor=NavyBlue,urlcolor=DarkRed,filecolor=green,bookmarksopen=true]{hyperref} % links coloridos
\usepackage[all]{hypcap}                % soluciona o problema com o hyperref e capitulos
\usepackage[square,sort,nonamebreak,comma]{natbib}  % citação bibliográfica alpha (alpha-ime.bst)
\fontsize{60}{62}\usefont{OT1}{cmr}{m}{n}{\selectfont}

% ---------------------------------------------------------------------------- %
% Cabeçalhos similares ao TAOCP de Donald E. Knuth
\usepackage{fancyhdr}
\pagestyle{fancy}
\fancyhf{}
\renewcommand{\chaptermark}[1]{\markboth{\MakeUppercase{#1}}{}}
\renewcommand{\sectionmark}[1]{\markright{\MakeUppercase{#1}}{}}
\renewcommand{\headrulewidth}{0pt}

% ---------------------------------------------------------------------------- %
\graphicspath{{./figuras/}}             % caminho das figuras (recomendável)
\frenchspacing                          % arruma o espaço: id est (i.e.) e exempli gratia (e.g.) 
\urlstyle{same}                         % URL com o mesmo estilo do texto e não mono-spaced
\makeindex                              % para o índice remissivo
\raggedbottom                           % para não permitir espaços extra no texto
\fontsize{60}{62}\usefont{OT1}{cmr}{m}{n}{\selectfont}
\cleardoublepage
\normalsize

% ---------------------------------------------------------------------------- %
% Opções de listing usados para o código fonte
% Ref: http://en.wikibooks.org/wiki/LaTeX/Packages/Listings
\lstset{ %
language=Java,                   % choose the language of the code
basicstyle=\footnotesize,       % the size of the fonts that are used for the code
numbers=left,                   % where to put the line-numbers
numberstyle=\footnotesize,      % the size of the fonts that are used for the line-numbers
stepnumber=1,                   % the step between two line-numbers. If it's 1 each line will be numbered
numbersep=5pt,                  % how far the line-numbers are from the code
showspaces=false,               % show spaces adding particular underscores
showstringspaces=false,         % underline spaces within strings
showtabs=false,                 % show tabs within strings adding particular underscores
frame=single,                   % adds a frame around the code
framerule=0.6pt,
tabsize=2,                      % sets default tabsize to 2 spaces
captionpos=b,                   % sets the caption-position to bottom
breaklines=true,                % sets automatic line breaking
breakatwhitespace=false,        % sets if automatic breaks should only happen at whitespace
escapeinside={\%*}{*)},         % if you want to add a comment within your code
backgroundcolor=\color[rgb]{1.0,1.0,1.0}, % choose the background color.
rulecolor=\color[rgb]{0.8,0.8,0.8},
extendedchars=true,
xleftmargin=10pt,
xrightmargin=10pt,
framexleftmargin=10pt,
framexrightmargin=10pt
}

% ---------------------------------------------------------------------------- %
% Corpo do texto
\begin{document}
\frontmatter 
% cabeçalho para as páginas das seções anteriores ao capítulo 1 (frontmatter)
\fancyhead[RO]{{\footnotesize\rightmark}\hspace{2em}\thepage}
\setcounter{tocdepth}{2}
\fancyhead[LE]{\thepage\hspace{2em}\footnotesize{\leftmark}}
\fancyhead[RE,LO]{}
\fancyhead[RO]{{\footnotesize\rightmark}\hspace{2em}\thepage}

\onehalfspacing  % espaçamento

% ---------------------------------------------------------------------------- %
% CAPA
% Nota: O título para as dissertações/teses do IME-USP devem caber em um 
% orifício de 10,7cm de largura x 6,0cm de altura que há na capa fornecida pela SPG.
\thispagestyle{empty}
\begin{center}
    \vspace*{2.3cm}
    \textbf{
        \Large{Utilização de sistemas distribuídos em jogos eletrônicos livres}
    }\\
    
    \vspace*{1.2cm}
    \Large{Henrique Gemignani Passos Lima}
    
    \vskip 2cm
    \textsc{Trabalho de conclusão de curso} 
    
    \vskip 10cm
    Orientador: Daniel Macêdo Batista

   	\vskip 3cm
    
    \normalsize{São Paulo, Dezembro de 2013}
\end{center}

\pagenumbering{roman}     % começamos a numerar 

% --------------------------------- %
% Agradecimentos
\chapter*{Agradecimentos}

TODO

% ---------------------------------------------------------------------------- %
% Resumo
\chapter*{Resumo}

Desenvolvedores de jogos eletrônicos livres que desejam implementar alguma
funcionalidade multijogador sofrem com um problema: manter servidores, ou pelo
menos uma lista desses. Jogos comerciais mais antigos para PC adotavam a
estratégia de permitir que o jogador se conectasse a um endereço IP qualquer.
Também era popular o jogo ter um lista de servidores oficial, que podia ser
acessada de dentro do jogo. Já jogos comerciais recentes sempre possuem um
servidor oficial principal do qual todas funcionalidades multijogador dependem.
Não é comum um jogo moderno ter suporte a se conectar a um endereço IP qualquer.
\\
\\
No caso de um jogo livre, o desenvolvedor tem duas opções: manter ele mesmo um
servidor principal ou deixar esse trabalho para a comunidade. O primeiro implica
num custo de manutenção para o desenvolvedor, enquanto o segundo aumenta a
barreira de entrada para novos jogadores, dado que o jogador precisa ativamente
procurar tal lista.
\\
\\
Neste trabalho, será implementado uma rede distribuída descentralizada na qual
todos os jogadores e servidores participam. Essa implementação será realizada no
jogo \textit{vikings}, um jogo eletrônico de plataforma 2D que narra a aventura
de um viking que almeja se tornar o líder de sua vila, e foi desenvolvido pelo
USPGameDev, um grupo de pesquisa e desenvolvimento de jogos da Universidade de
São Paulo.

% ------------------------------------------------------------------ %
% facilidades                                                        %
% ------------------------------------------------------------------ %

\def\cyclic#1{\langle #1 \rangle}

% comandos novos %
\newcounter{defcnt}
\newcommand\definicao[2]{
    \stepcounter{defcnt}
    \vspace{0.5cm}
    \vbox{
    \textbf{Definição \thedefcnt \hspace{1cm} #1} \\ #2
    \begin{center}
        $ \square $
    \end{center}
}
}
\newcommand\notacao{
    \textbf{Notação} \hspace{0.2cm}
}

% ---------------------------------------------------------------------------- %
% Sumário
\tableofcontents    % imprime o sumário

% ---------------------------------------------------------------------------- %
% Capítulos do trabalho
\mainmatter

% cabeçalho para as páginas de todos os capítulos
\fancyhead[RE,LO]{\thesection}

\singlespacing              % espaçamento simples

\input cap-objetiva
\input cap-subjetiva

% cabeçalho para os apêndices
\renewcommand{\chaptermark}[1]{\markboth{\MakeUppercase{\appendixname\ \thechapter}} {\MakeUppercase{#1}} }
\fancyhead[RE,LO]{}
\appendix

% ---------------------------------------------------------------------------- %
% Bibliografia
\renewcommand\bibname{Referências}
\backmatter \singlespacing   % espaçamento simples
\bibliographystyle{alpha-ime}% citação bibliográfica alpha
\bibliography{bibliografia}  % associado ao arquivo: 'bibliografia.bib'

\nocite{*}

\end{document}
