%% ------------------------------------------------------------------------- %%
\chapter*{Parte Subjetiva}
\label{sec:parte_subjetiva}

\chapter{Desafios e Frustrações}
\label{sec:desafios_frustracoes}

  \begin{enumerate}
    \item Problemas em como começar a organizar a monografia, por falta de experiencia em
      usar \LaTeX{} para algo além de alguns slides.
    
    \item O sistemas de threads da LÖVE é bem limitado, forçando a implementação do cliente da rede
      a usar as co-rotinas do Lua, que é um sistema de \textit{fluxos de execução colaborativos}.
      Mesmo assim, operações que travam o fluxo de operações continuam causando problemas.
    
    \item Gastar mais de um mês tentando prever todas as necessidades do protocolo e ficar travado nisso.
    
    \item NAT atrapalha tudo. Não consegui testar em nenhum ambiente complexo como, por exemplo,
      utilizar o Chat através da Internet com amigos.

    \item Problemas de conflito de horário da matéria de Programação para Redes de Computadores com
      a minha grade horária, em particular com Álgebra II e Sistemas Operacionais, que me impediram de
      fazer essa matéria por dois anos seguidos...
  \end{enumerate}

\chapter{Relação entre o trabalho de formatura e disciplinas do BCC}
\label{sec:relacao_disciplinas_bcc}
\newcommand\materia[3]{\noindent \textbf{#1} - \texttt{#2}\\\indent #3\vspace{0.5cm}\\}

\materia{MAC0122}{Princípios de Desenvolvimento de Algoritmos}{
  Aprendi a programar em C e a gerenciar memória! Um começo para tudo que veio depois :)
}
\materia{MAC0342}{Laboratório de Programação Extrema}{
  Matéria que melhor ensinou os princípios da metodologia ágil, além de fornecer um ambiente
  amigável para trabalhar em projetos grandes e estabelecidos.
}
\materia{MAC0316}{Conceitos Fundamentais de Linguagens de Programação}{
  Introduziu o paradigma de programação funcional, que traz conceitos úteis mesmo quando não se
  utiliza uma linguagem de programação funcional.
}
\materia{MAC0448}{Programação para Redes de Computadores}{
  Mesmo tendo feito um semestre depois de começar o TCC, experiências que tive durante a matéria
  me ajudaram a organizar o código do TCC.
}
\materia{USPGD001}{USPGameDev}{
  Não é exatamente uma disciplina, mas o USPGameDev foi a parte da minha vida acadêmica que teve
  a maior influência no meu aprendizado. No começo, havia veteranos para nos guiar. Ao longo dos
  anos, tivemos experiência com softwares de controle de versão, portabilidade de código,
  fazendo lançamentos, apresentações.

  PS: A sigla dessa 'matéria' veio do fórum do paca\footnotemark{} do grupo.
}

\footnotetext{
   Link: \url{http://paca.ime.usp.br/course/view.php?id=431} (Última visita: 02/12/2013)
}

\chapter{Próximos passos}
\label{sec:proximos_passos}

Pretendo continuar com o projeto, seguindo o plano descrito na seção \ref{sec:conclusoes:futuro}.
No USPGameDev, planejamos começar um novo projeto em Dezembro de 2013 que utilizará o Etherclan para
a parte multijogador, dessa vez planejando multijogador desde o princípio.




