\chapter{Rede Etherclan}
\label{sec:etherclan}

Este capítulo apresenta a principal contribuíção deste TCC, a rede Etherclan.

\section{Arquitetura da Rede}
\label{sec:etherclan:rede}

  A rede Etherclan é composta por diversos nós conectados diretamente um nos outros, sem a necessidade
  de nenhum servidor centralizado.
  
  Para um programa usufruir dos recursos da rede, é necessário primeiro fazer parte da rede. Para 
  tal duas coisas são necessárias:
  
  \begin{itemize}
    \item Conhecer pelo menos 1 nó membro da rede.
    \item Criar um nó para inserir na rede.
  \end{itemize}
  
  \subsection{Um Nó}
    \subsubsection{O que é um nó}
      Todo nó deve possuir um UUIDv4 próprio.
      Para um nó ser considerado ativo, deve-se escutar em alguma porta TCP qualquer e responder às
      mensagens recebidas. Identificadores universalmente únicos (\textit{Universally unique identifier},
      UUID), definidos pelo RFC4122 \cite{rfc4122}, são números de 128 bits construídos de maneira a
      serem praticamente únicos.
      
    \subsubsection{Processo de criação de um nó ativo}
      Deve-se gerar um UUIDv4, e criar um socket com a opção de \textit{listen} habilitada.

    \subsubsection{Adicionar o nó na rede}
      Com o nó ativo, basta enviar a mensagem 'ANNOUNCE\_SELF' para pelo menos 1 nó que faça parte da rede.

\section{Protocolo}
\label{sec:etherclan:protocolo}
  Na rede Etherclan, as mensagens trocadas pelos nós são cadeias de caracteres codificadas em UTF-8,
  terminadas por uma quebra de linha (caractere ASCII '\textbackslash n'). Uma mensagem pode ter 0
  ou mais ocorrências do caractere  ASCII~'~', dividindo a mensagem em um comando seguido de 0 ou
  mais argumentos.

  \subsection{Comandos}
  \label{sec:etherclan:protocolo:comandos}
    Temos a seguir todos os comandos da rede, incluindo quais argumentos são esperados
    para cada um.
      
    \begin{itemize}
      \item ANNOUNCE\_SELF <UUID> <PORT> [<SERVICE1> [<SERVICE2> [...]]] \\
        O computador que envia este comando está anunciando que criou um novo nó, com o UUID enviado,
        que está escutando no endereço IP que utilizou para enviar a mensagem, na porta <PORT> e que
        conhece os serviços <SERVICE1>, <SERVICE2>, ...
        
        Esse comando não possui resposta.
    
      \item SERVICE <NAME> ... \\
        Comando que deve ser tratado utilizando o serviço de nome <NAME>.
        
        Serviços são extensões da rede que podem fornecer recursos adicionais, além de servirem
        para identificar a qual software o nó pertence.
        A resposta desse comando é definido pelo serviço.
    
      \item REQUEST\_NODE\_LIST \\
        Requisição por informações de outros nós que pertencem à rede.
        
        Sempre que um nó A receber esse comando, ele deve responder com diversos 'NODE\_INFO'.
        Quantos e sobre quais nós fica a critério da implementação.
        
      \item KEEP\_ALIVE <ON ou OFF> \\
        KEEP\_ALIVE ON indica que a conexão deve ser mantida aberta com o propósito de enviar
        diversos comandos.
        KEEP\_ALIVE OFF indica que a conexão deve ser fechada.
        
    \end{itemize}
  
  \subsection{Respostas}
    Temos a seguir todos as respostas que podem ser enviadas às mensagens apresentadas na subseção
    \ref{sec:etherclan:protocolo:comandos}.
    
    \begin{itemize}
      \item UNKNOWN\_COMMAND \\
        Enviada após receber algum comando desconhecido.
        
      \item INPUT\_ERROR \\
        Enviada após receber um número incorreto de argumentos para algum comando, ou algum
        argumento do tipo errado.
    
      \item NODE\_INFO <UUID> <IP> <PORT> [<SERVICE1> [<SERVICE2> [...]]] \\
        Informa que existe um nó com uuid <UUID>, que conhece os serviços <SERVICE1>, <SERVICE2>, ...,
        e está escutando em <IP>:<PORT>
          
    \end{itemize}
  
\section{Implementação}
  Uma implementação de exemplo foi realizada utilizando a linguagem de programação Lua, com
  a biblioteca LuaSocket\footnotemark. A escolha por Lua se deve para facilitar a integração
  com o \textit{vikings}. A implementação está disponível em \url{https://github.com/henriquegemignani/etherclan}.
  
  \footnotetext{
    Página do projeto: \url{http://luasocket.luaforge.net/} (última visita em 23/11/2013).
  }

\section{Problemas}
  \subsection{Bootstrap}
    Dado o acesso a um conjunto de nós, é possível conseguir ainda mais nós para adicionar a esse
    conjunto, permitindo que o processo continue. Logo, para iniciar o processo, é necessário
    conhecer pelo menos 1 nó, como descrito na seção \ref{sec:etherclan:rede}. Isso é conhecido como
    bootstrapping \cite{tech:bootstrapping}.
    
    Nisso, chegamos num problema: como conseguir esse primeiro nó?
    
    Não foi possível achar nenhuma solução ótima para esse problema, nem mesmo achar como que outras
    redes similares lidam com esse problema.
    
    A solução adotada na implementação de exemplo foi requisitar do software que utilizar a rede uma
    lista de nós. Nos casos de uso, essas listas eram fixas.
    
  \subsection{Estratégias de busca e gerenciamento de nós conhecidos}
    A ideia original era que o gerenciamento de nós conhecidos, assim como critérios de como e
    quando realizar as buscas era de responsabilidade da implementação, pois acreditava-se que não
    seria tão importante ter bem definido essa parte, e deixando para implementações decidirem
    por algoritmos elaborados.
    
    Durante a implementação da primeira prova de conceito, tornou-se óbvio que sem bons critérios
    escolhidos nessas áreas, a rede pode chegar a não funcionar. Alguns problemas:
    \begin{enumerate}
      \item Sobre quando e como realizar buscas, é possível que alguns nós não sejam encontrados e
        que se gaste muito tempo realizando buscas desnecessárias.
      \item Sobre quando remover um nó, um nó inacessível degrada o desempenho das outras operações,
        enquanto esquecer nós demais pode tornar a rede desconexa.
    \end{enumerate}